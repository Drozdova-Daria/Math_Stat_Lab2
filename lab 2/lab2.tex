\documentclass{article}

\usepackage[utf8]{inputenc}
\usepackage[russian]{babel}
\usepackage{graphicx}
\graphicspath{{pictures/}}
\DeclareGraphicsExtensions{.pdf,.png,.jpg}
\usepackage[unicode, pdftex]{hyperref}
\usepackage{csvsimple}

\begin{document}

\begin{titlepage}
  \thispagestyle{empty}
  \centerline {Санкт-Петербургский политехнический университет}
  \centerline { им. Петра Великого}
  \centerline { }
  \centerline {Институт прикладной математики и механики} 
  \centerline {Кафедра "Прикладная математика"}
  \vfill
  \centerline{\textbf{Отчёт}}
  \centerline{\textbf{по лабораторной работе №2}}
  \centerline{\textbf{по дисциплине}}
  \centerline{\textbf{"Математическая статистика"}}
  \vfill
  \hfill
  \begin{minipage}{0.45\textwidth}
  Выполнил студент:\\
  Дроздова Дарья Александровна\\
  группа: 3630102/80401 \\
  \\
  Проверил:\\
  к.ф.-м.н., доцент \\
  Баженов Александр Николаевич
  \end{minipage}
  \vfill
  \centerline {Санкт-Петербург}   
  \centerline {2021 г.}  
\end{titlepage}

\newpage
\setcounter{page}{2}
\tableofcontents

\newpage
\listoftables

\newpage
\section{Постановка задачи}
  Для 5 распределений:
  \begin{itemize}
    \item Нормальное распределение: $N(x,0,1)$
    \item Распределение Коши: $C(x,0,1)$
    \item Распределенеие Лапласа: $L(x,0,\sqrt{2})$
    \item Распределение Пуассона: $P(k,10)$
    \item Равномерное распределение: $U(x,-\sqrt{3}, \sqrt{3})$
  \end{itemize}
  Сгенерировать выборки мощностью 10, 100 и 1000 элементов. Для каждой выборки вычислить статистические положения данных: 
  \begin{itemize}
    \item Выборочное среднее - $\overline{x}$
    \item Выборочная медиана - $med$ $x$
    \item Полусумма экстремальных выборочных элементов - $z_R$
    \item Полусумма квартилей - $z_Q$
    \item Усеченое среднее - $z_{tr}$
  \end{itemize}
  Повторить вычисления 1000 раз для каждой выборки и найти среднее характеристик положения и их квадратов:
  \begin{equation}
    E(z)=\overline{z}
    \label{eq: 1}
  \end{equation}
  Вычислить оценку дисперсии по формуле:
  \begin{equation}
    D(z)=\overline{z^2} - \overline{z}^2
    \label{eq: 2}
  \end{equation}
  Полученные данные представить в виде таблиц
  
\newpage
\section{Теория}
\subsection{Выриационный ряд}
  Вариационный ряд - последовательность элементов выборки, расположенных в неубывающем порядке. Одинаковые элементы повторяются. Запись вариационного ряда: $x_{(1)}, x_{(2)}, ...,x_{(n)}$
\subsection{Выборочное среднее}
 Выборочным средним называется среднее арифметическое значений вариант в выборке, определяется по формуле:
 \begin{equation}
 \overline{x} = \frac{1}{n}\sum_{i=1}^{n}x_i
 \label{eq: 3}
 \end{equation}
 Выборочное среднее используется для статистической оценки математического ожидания исследуемой случайной величины.
\subsection{Выборочная медиана}
  Выборочная медиана определяется по формуле:
  \begin{equation}
  med \; x=\left \{\ 
  \begin{array}{rcl}
  x_{(l+1)}& \qquad \mbox{при} \; n=2l+1 \\
  \frac{x_{(l)}+x_{(l+1)}}{2}& \qquad \mbox{при} \; n=2l
  \end{array}\right.
  \label{eq: 4}
  \end{equation}
\subsection{Полусумма экстремальных выборочных элементов}
  Полусумма экстремальных выборочных элементов определяется по формуле:
  \begin{equation}
  z_R=\frac{x_{(1)}-x_{(n)}}{2}
  \label{eq: 5}
  \end{equation}
\subsection{Полусумма квартилей}
  Для вычисления полусуммы квартилей необходимо для начала вычислить выборочную квартиль $z_p$ порядка $p$, которая определяется по формуле
  \begin{equation}
   z_p=\left \{\ 
  \begin{array}{rcl}
  x_{([np]+1)}& \qquad \mbox{при} \; np \; \mbox{дробном} \\
  x_{(np)}& \qquad \mbox{при} \; np \; \mbox{целом}
  \end{array}\right.
  \label{eq: 6}
  \end{equation}
  Полусумма квартилей определяется по формуле
  \begin{equation}
  z_Q=\frac{z_{\frac{1}{4}} + z_{\frac{3}{4}}}{2}
  \label{eq: 7}
  \end{equation}
\subsection{Усечённое среднее}
  Формула усечённого среднего:
  \begin{equation}
  z_{tr} = \frac{1}{n-2r}\sum_{t=r+1}^{n-r}x_{(i)}, \; r\approx\frac{n}{4}
  \label{eq: 8}
  \end{equation}

\newpage
\section{Реализация}
Лабораторная работа выполнена на языке программирования Python в среде разработки PyCharm с использованием библиотек: numpy, skipy для построения выборок.
\\
\\
Код программы расположен в репозитории GitHub по ссылке: \url{https://github.com/Drozdova-Daria/Math_Stat_Lab2}

\newpage
\section{Результаты}
\subsection{Нормальное распределение}
\begin{table}[hb]
\begin{center}
\begin{tabular}{|c|c|c|c|c|c|}
\hline 
\multicolumn{6}{|c|}{Нормальное распределение $n=10$} \\ 
\hline 
  & $\overline{x}$ (\ref{eq: 3}) & $med \; x$ (\ref{eq: 4}) & $z_R$ (\ref{eq: 5}) & $z_Q$ (\ref{eq: 7}) & $z_{tr}$ (\ref{eq: 8}) \\ 
\hline 
$E(z)$ (\ref{eq: 1}) & 0.00031 & -0.00939 & -0.0029 & 0.15492 & -0.10765 \\ 
\hline 
$D(z)$ (\ref{eq: 2}) & 0.89814 & 0.4025 & 1.73983 & 1.02643 & 0.31975 \\ 
\hline 
$E(z)-\sqrt{D(z)}$ & -0.94739 & -0.64381 & -1.32193 & -0.85821 & -0.67312 \\ 
\hline 
$E(z)+\sqrt{D(z)}$ & 0.94801 & 0.62504 & 1.31612 & 1.16805 & 0.45781 \\ 
\hline
$Estimation (\widehat{E}(x))$ & 0 & 0 & 0 & 0 & 0 \\
\hline
\end{tabular} 
\caption{Нормальное распределение для $n=10$}
\end{center}
\end{table}

\begin{table}[hb]
\begin{center}
\begin{tabular}{|c|c|c|c|c|c|}
\hline 
\multicolumn{6}{|c|}{Нормальное распределение $n=100$} \\ 
\hline 
  & $\overline{x}$ (\ref{eq: 3}) & $med \; x$ (\ref{eq: 4}) & $z_R$ (\ref{eq: 5}) & $z_Q$ (\ref{eq: 7}) & $z_{tr}$ (\ref{eq: 8}) \\ 
\hline 
$E(z)$ (\ref{eq: 1}) & -0.00048 & 0.00325 & -0.00292 & -0.01187 & -0.0137 \\ 
\hline 
$D(z)$ (\ref{eq: 2}) & 0.99702 & 0.45735 & 3.74868 & 0.69462 & 0.50724 \\ 
\hline 
$E(z)-\sqrt{D(z)}$ & -0.99899 & -0.67302 & -1.93907 & -0.84531 & -0.72591 \\ 
\hline 
$E(z)+\sqrt{D(z)}$ & 0.99803 & 0.67953 & 1.93323 & 0.82157 & 0.6985 \\ 
\hline 
$Estimation (\widehat{E}(x))$ & 0 & 0 & 0 & 0 & 0 \\
\hline
\end{tabular} 
\caption{Нормальное распределение для $n=100$}
\end{center}
\end{table}

\begin{table}[hb]
\begin{center}
\begin{tabular}{|c|c|c|c|c|c|}
\hline 
\multicolumn{6}{|c|}{Нормальное распределение $n=1000$} \\ 
\hline 
  & $\overline{x}$ (\ref{eq: 3}) & $med \; x$ (\ref{eq: 4}) & $z_R$ (\ref{eq: 5}) & $z_Q$ (\ref{eq: 7}) & $z_{tr}$ (\ref{eq: 8}) \\ 
\hline 
$E(z)$ (\ref{eq: 1}) & 0.0002 & -0.00113 & -0.01873 & -0.00218 & -0.00125 \\ 
\hline 
$D(z)$ (\ref{eq: 2}) & 0.99653 & 0.45259 & 5.8749 & 0.71095 & 0.53241 \\ 
\hline 
$E(z)-\sqrt{D(z)}$ & -0.99807 & -0.67388 & -2.44254 & -0.84536 & -0.73092 \\ 
\hline 
$E(z)+\sqrt{D(z)}$ & 0.99846 & 0.67162 & 2.40509 & 0.841 & 0.72842 \\ 
\hline 
$Estimation (\widehat{E}(x))$ & 0 & 0 & 0 & 0 & 0 \\
\hline
\end{tabular} 
\caption{Нормальное распределение для $n=1000$}
\end{center}
\end{table}

\newpage
\subsection{Распределение Коши}

\begin{table}[hb]
\begin{center}
\begin{tabular}{|c|c|c|c|c|c|}
\hline 
\multicolumn{6}{|c|}{Распределение Коши $n=10$} \\ 
\hline 
  & $\overline{x}$ (\ref{eq: 3}) & $med \; x$ (\ref{eq: 4}) & $z_R$ (\ref{eq: 5}) & $z_Q$ (\ref{eq: 7}) & $z_{tr}$ (\ref{eq: 8}) \\ 
\hline 
$E(z)$ (\ref{eq: 1}) & -3.82847 & -0.01129 & -3.98716 & 1.04591 & -0.16992 \\ 
\hline 
$D(z)$ (\ref{eq: 2}) & 1739321.17684 & 1.46085 & 26995.15408 & 90.69031 & 1.35772 \\ 
\hline 
$E(z)-\sqrt{D(z)}$ & -1322.66173 & -1.21995 & -168.28918 & -8.47723 & -1.33513 \\ 
\hline 
$E(z)+\sqrt{D(z)}$ & 1315.0048 & 1.19737 & 160.31486 & 10.56906 & 0.99529 \\ 
\hline 
$Estimation (\widehat{E}(x))$ & - & 0 & - & - & 0 \\
\hline
\end{tabular} 
\caption{Распределение Коши для $n=10$}
\end{center}
\end{table}

\begin{table}[hb]
\begin{center}
\begin{tabular}{|c|c|c|c|c|c|}
\hline 
\multicolumn{6}{|c|}{Распределение Коши $n=100$} \\ 
\hline 
  & $\overline{x}$ (\ref{eq: 3}) & $med \; x$ (\ref{eq: 4}) & $z_R$ (\ref{eq: 5}) & 
$z_Q$ (\ref{eq: 7}) & $z_{tr}$ (\ref{eq: 8}) \\ 
\hline 
$E(z)$ (\ref{eq: 1}) & -3.82847 & 0.00044 & 2.62929 & -0.03175 & -0.01901 \\ 
\hline 
$D(z)$ (\ref{eq: 2}) & 1739321.17684 & 1.01715 & 534361.37129 & 3.05794 & 1.48688 \\ 
\hline 
$E(z)-\sqrt{D(z)}$ & -1322.66173 & -1.0081 & -728.37096 & -1.78045 & -1.23839 \\ 
\hline 
$E(z)+\sqrt{D(z)}$ & 1315.0048 & 1.00898 & 733.62954 & 1.71695 & 1.20037 \\ 
\hline 
$Estimation (\widehat{E}(x))$ & - & 0 & - & 0 & 0 \\
\hline
\end{tabular} 
\caption{Распределение Коши для $n=100$}
\end{center}
\end{table}

\begin{table}[hb]
\begin{center}
\begin{tabular}{|c|c|c|c|c|c|}
\hline 
\multicolumn{6}{|c|}{Распределение Коши $n=1000$} \\ 
\hline 
  & $\overline{x}$ (\ref{eq: 3}) & $med \; x$ (\ref{eq: 4}) & $z_R$ (\ref{eq: 5}) & $z_Q$ (\ref{eq: 7}) & $z_{tr}$ (\ref{eq: 8}) \\ 
\hline 
$E(z)$ (\ref{eq: 1}) & 15.44914 & 0.00096 & -537.58428 & 0.00031 & -0.00543 \\ 
\hline 
$D(z)$ (\ref{eq: 2}) & 255019222.11428 & 1.01182 & 542910442.97336 & 2.9883 & 1.53436 \\ 
\hline 
$E(z)-\sqrt{D(z)}$ & -15953.87213 & -1.00494 & -23838.02296 & -1.72836 & -1.24412 \\ 
\hline 
$E(z)+\sqrt{D(z)}$ & 15984.77042 & 1.00685 & 22762.85441 & 1.72898 & 1.23326 \\ 
\hline 
$Estimation (\widehat{E}(x))$ & - & 0 & - & 0 & 0 \\
\hline
\end{tabular} 
\caption{Распределение Коши для $n=1000$}
\end{center}
\end{table}

\newpage
\subsection{Распределение Лапласа}

\begin{table}[hb]
\begin{center}
\begin{tabular}{|c|c|c|c|c|c|}
\hline 
\multicolumn{6}{|c|}{Распределение Лапласа $n=10$} \\ 
\hline 
  & $\overline{x}$ (\ref{eq: 3}) & $med \; x$ (\ref{eq: 4}) & $z_R$ (\ref{eq: 5}) & $z_Q$ (\ref{eq: 7}) & $z_{tr}$ (\ref{eq: 8}) \\ 
\hline 
$E(z)$ (\ref{eq: 1}) & -0.00455 & -0.01221 & -0.02349 & 0.39825 & -0.18219 \\ 
\hline 
$D(z)$ (\ref{eq: 2}) & 3.68578 & 1.08921 & 8.28112 & 3.77595 & 0.83721 \\ 
\hline 
$E(z)-\sqrt{D(z)}$ & -1.92439 & -1.05587 & -2.90118 & -1.54493 & -1.09718 \\ 
\hline 
$E(z)+\sqrt{D(z)}$ & 1.91529 & 1.03144 & 2.8542 & 2.34143 & 0.7328 \\ 
\hline 
$Estimation (\widehat{E}(x))$ & 0 & 0 & 0 & 0 & 0 \\
\hline
\end{tabular} 
\caption{Распределение Лапласа для $n=10$}
\end{center}
\end{table}

\begin{table}[hb]
\begin{center}
\begin{tabular}{|c|c|c|c|c|c|}
\hline 
\multicolumn{6}{|c|}{Распределение Лапласа $n=100$} \\ 
\hline 
  & $\overline{x}$ (\ref{eq: 3}) & $med \; x$ (\ref{eq: 4}) & $z_R$ (\ref{eq: 5}) & $z_Q$ (\ref{eq: 7}) & $z_{tr}$ (\ref{eq: 8}) \\ 
\hline 
$E(z)$ (\ref{eq: 1}) & -0.0024 & -0.00183 & 0.0736 & -0.03431 & -0.02274 \\ 
\hline 
$D(z)$ (\ref{eq: 2}) & 3.90323 & 0.96503 & 27.82659 & 1.96569 & 1.21596 \\ 
\hline 
$E(z)-\sqrt{D(z)}$ & -1.97806 & -0.98419 & -5.20149 & -1.43634 & -1.12545 \\ 
\hline 
$E(z)+\sqrt{D(z)}$ & 1.97326 & 0.98053 & 5.34869 & 1.36771 & 1.07996 \\ 
\hline 
$Estimation (\widehat{E}(x))$ & 0 & 0 & 0 & 0 & 0 \\
\hline
\end{tabular} 
\caption{Распределение Лапласа для $n=100$}
\end{center}
\end{table}

\begin{table}[hb]
\begin{center}
\begin{tabular}{|c|c|c|c|c|c|}
\hline 
\multicolumn{6}{|c|}{Распределение Лапласа $n=1000$} \\ 
\hline 
  & $\overline{x}$ (\ref{eq: 3}) & $med \; x$ (\ref{eq: 4}) & $z_R$ (\ref{eq: 5}) & $z_Q$ (\ref{eq: 7}) & $z_{tr}$ (\ref{eq: 8}) \\ 
\hline 
$E(z)$ (\ref{eq: 1}) & -0.00507 & -0.00072 & -0.04782 & -0.00176 & -0.00275 \\ 
\hline 
$D(z)$ (\ref{eq: 2}) & 4.00349 & 0.96406 & 55.81322 & 2.00036 & 1.27229 \\ 
\hline 
$E(z)-\sqrt{D(z)}$ & -2.00594 & -0.98258 & -7.51865 & -1.4161 & -1.13071 \\ 
\hline 
$E(z)+\sqrt{D(z)}$ & 1.9958 & 0.98115 & 7.423 & 1.41258 & 1.12521 \\ 
\hline 
$Estimation (\widehat{E}(x))$ & 0 & 0 & 0 & 0 & 0 \\
\hline
\end{tabular} 
\caption{Распределение Лапласа для $n=1000$}
\end{center}
\end{table}

\newpage
\subsection{Распределение Пуассона}
\begin{table} [hb]
\begin{center}
\begin{tabular}{|c|c|c|c|c|c|}
\hline 
\multicolumn{6}{|c|}{Распределение Пуассона $n=10$} \\ 
\hline 
  & $\overline{x}$ (\ref{eq: 3}) & $med \; x$ (\ref{eq: 4}) & $z_R$ (\ref{eq: 5}) & 
$z_Q$ (\ref{eq: 7}) & $z_{tr}$ (\ref{eq: 8}) \\ 
\hline 
$E(z)$ (\ref{eq: 1}) & 9.9699 & 9.8925 & 10.2735 & 10.467 & 9.86033 \\ 
\hline 
$D(z)$ (\ref{eq: 2}) & 8.86209 & 0.32775 & 24.06775 & 7.872 & 13.90278 \\ 
\hline 
$E(z)-\sqrt{D(z)}$ & 6.99297 & 9.32001 & 5.36761 & 7.66129 & 6.13169 \\ 
\hline 
$E(z)+\sqrt{D(z)}$ & 12.94683 & 10.46499 & 15.17939 & 13.57271 & 11.58898 \\ 
\hline
$Estimation (\widehat{E}(x))$ & $9^{+1}_{-1}$ & $9^{+1}_{-1}$ & $10^{+1}_{-1}$ & $10^{+1}_{-1}$ & $9^{+1}_{-1}$ \\
\hline 
\end{tabular} 
\caption{Распределение Пуассона для $n=10$}
\end{center}
\end{table}

\begin{table} [hb]
\begin{center}
\begin{tabular}{|c|c|c|c|c|c|}
\hline 
\multicolumn{6}{|c|}{Распределение Пуассона $n=100$} \\ 
\hline 
  & $\overline{x}$ (\ref{eq: 3}) & $med \; x$ (\ref{eq: 4}) & $z_R$ (\ref{eq: 5}) & $z_Q$ (\ref{eq: 7}) & $z_{tr}$ (\ref{eq: 8}) \\ 
\hline 
$E(z)$ (\ref{eq: 1}) & 9.98856 & 9.846 & 10.937 & 9.87 & 9.5961 \\ 
\hline 
$D(z)$ (\ref{eq: 2}) & 9.93849 & 0.0185 & 62.0415 & 4.529 & 3.33149 \\ 
\hline 
$E(z)-\sqrt{D(z)}$ & 6.83602 & 9.70999 & 3.06036 & 7.74186 & 7.77086 \\ 
\hline 
$E(z)+\sqrt{D(z)}$ & 13.1411 & 9.98201 & 18.81364 & 11.99814 & 11.42134 \\ 
\hline 
$Estimation (\widehat{E}(x))$ & $9^{+1}_{-1}$ & $9^{+1}_{-1}$ & $10^{+1}_{-1}$ & $9^{+1}_{-1}$ & $9^{+1}_{-1}$ \\
\hline 
\end{tabular} 
\caption{Распределение Пуассона для $n=100$}
\end{center}
\end{table}

\begin{table}[hb]
\begin{center}
\begin{tabular}{|c|c|c|c|c|c|}
\hline 
\multicolumn{6}{|c|}{Распределение Лапласа $n=1000$} \\ 
\hline 
  & $\overline{x}$ (\ref{eq: 3}) & $med \; x$ (\ref{eq: 4}) & $z_R$ (\ref{eq: 5}) & $z_Q$ (\ref{eq: 7}) & $z_{tr}$ (\ref{eq: 8}) \\ 
\hline 
$E(z)$ (\ref{eq: 1}) & 9.99407 & 9.996 & 10.663 & 9.9925 & 9.83788 \\ 
\hline 
$D(z)$ (\ref{eq: 2}) & 9.97752 & 0.0 & 101.426 & 4.03375 & 1.72204 \\ 
\hline 
$E(z)-\sqrt{D(z)}$ & 6.83535 & 9.996 & 0.59195 & 7.98408 & 8.52562 \\ 
\hline 
$E(z)+\sqrt{D(z)}$ & 13.15279 & 9.996 & 20.73405 & 12.00092 & 11.15015 \\ 
\hline 
$Estimation (\widehat{E}(x))$ & $9^{+1}_{-1}$ & $9^{+1}_{-1}$ & $10^{+1}_{-1}$ & $9^{+1}_{-1}$ & $9^{+1}_{-1}$ \\
\hline 
\end{tabular} 
\caption{Распределение Пуассона для $n=1000$}
\end{center}
\end{table}

\newpage
\subsection{Равномерное распределение}

\begin{table}[hb]
\begin{center}
\begin{tabular}{|c|c|c|c|c|c|}
\hline 
\multicolumn{6}{|c|}{Равномерное распределение $n=10$} \\ 
\hline 
  & $\overline{x}$ (\ref{eq: 3}) & $med \; x$ (\ref{eq: 4}) & $z_R$ (\ref{eq: 5}) & $z_Q$ (\ref{eq: 7}) & $z_{tr}$ (\ref{eq: 8}) \\ 
\hline 
$E(z)$ (\ref{eq: 1}) & -0.86762 & -0.86979 & -0.86709 & -0.76789 & -0.77633 \\ 
\hline 
$D(z)$ (\ref{eq: 2}) & 0.22773 & 0.0113 & 0.51162 & 0.51544 & -0.047 \\ 
\hline 
$E(z)-\sqrt{D(z)}$ & -1.34482 & -0.97608 & -1.58236 & -1.48583 & -0.99313 \\ 
\hline 
$E(z)+\sqrt{D(z)}$ & -0.39041 & -0.7635 & -0.15182 & -0.04995 & -0.55953 \\ 
\hline 
$Estimation (\widehat{E}(x))$ & 0 & 0 & 0 & 0 & 0 \\
\hline
\end{tabular} 
\caption{Равномерное распределение для $n=10$}
\end{center}
\end{table}

\begin{table}[hb]
\begin{center}
\begin{tabular}{|c|c|c|c|c|c|}
\hline 
\multicolumn{6}{|c|}{Равномерное распределение для $n=100$} \\ 
\hline 
  & $\overline{x}$ (\ref{eq: 3}) & $med \; x$ (\ref{eq: 4}) & $z_R$ (\ref{eq: 5}) & $z_Q$ (\ref{eq: 7}) & $z_{tr}$ (\ref{eq: 8}) \\ 
\hline 
$E(z)$ (\ref{eq: 1}) & -0.86473 & -0.86417 & -0.85167 & -0.87358 & -0.85902 \\ 
\hline 
$D(z)$ (\ref{eq: 2}) & 0.24768 & 0.00016 & 0.71973 & 0.15428 & 0.04585 \\ 
\hline 
$E(z)-\sqrt{D(z)}$ & -1.3624 & -0.87668 & -1.7143 & -1.26636 & -1.07316 \\ 
\hline 
$E(z)+\sqrt{D(z)}$ & -0.36705 & -0.85167 & -0.01756 & -0.4808 & -0.64488 \\ 
\hline
$Estimation (\widehat{E}(x))$ & 0 & 0 & 0 & 0 & 0 \\
\hline 
\end{tabular} 
\caption{Равномерное распределение для $n=100$}
\end{center}
\end{table}

\begin{table}[hb]
\begin{center}
\begin{tabular}{|c|c|c|c|c|c|}
\hline 
\multicolumn{6}{|c|}{Равномерное распределение $n=1000$} \\ 
\hline 
  & $\overline{x}$ (\ref{eq: 3}) & $med \; x$ (\ref{eq: 4}) & $z_R$ (\ref{eq: 5}) & $z_Q$ (\ref{eq: 7}) & $z_{tr}$ (\ref{eq: 8}) \\ 
\hline 
$E(z)$ (\ref{eq: 1}) & -0.86623 & -0.86746 & -0.86602 & -0.86691 & -0.86539 \\ 
\hline 
$D(z)$ (\ref{eq: 2}) & 0.24937 & 0.0 & 0.74697 & 0.18458 & 0.06108 \\ 
\hline 
$E(z)-\sqrt{D(z)}$ & -1.3656 & -0.86871 & -1.7303 & -1.29654 & -1.11254 \\ 
\hline 
$E(z)+\sqrt{D(z)}$ & -0.36686 & -0.86621 & -0.00175 & -0.43729 & -0.61824 \\ 
\hline 
$Estimation (\widehat{E}(x))$ & 0 & 0 & 0 & 0 & 0 \\
\hline
\end{tabular} 
\caption{Равномерное распределение для $n=1000$}
\end{center}
\end{table}

\newpage
\section{Обсуждение}
По результатам полученных данных можно заметить, что дисперсия характеристик распределения для распределения Коши дает большие значения даже при увеличении можщности выборки, это может быть связано с выбросами, которые наблюдались на гистограмме для данного распределения при выполнении лабораторной работы № 1.

\newpage
\begin{thebibliography}{4}
\addcontentsline{toc}{section}{\bibname}
\bibitem{cauchy}
Документация бибилиотеки scipy.stats.cauchy. 
\\ URL: https://docs.scipy.org/doc/scipy/reference/generated/scipy.stats.cauchy.html
\bibitem{laplace}
Документация бибилиотеки scipy.stats.laplace. 
\\ URL: https://docs.scipy.org/doc/scipy/reference/generated/scipy.stats.laplace.html
\bibitem{poisson}
Документация бибилиотеки scipy.stats.poisson.
\\ URL: https://docs.scipy.org/doc/scipy/reference/generated/scipy.stats.poisson.html
\bibitem{uniform}
Документация бибилиотеки scipy.stats.uniform.
\\ URL: https://docs.scipy.org/doc/scipy/reference/generated/scipy.stats.uniform.html
\end{thebibliography}

\end{document}
