\documentclass{article}

\usepackage[utf8]{inputenc}
\usepackage[russian]{babel}
\usepackage{graphicx}
\graphicspath{{pictures/}}
\DeclareGraphicsExtensions{.pdf,.png,.jpg}
\usepackage[unicode, pdftex]{hyperref}
\usepackage{csvsimple}

\begin{document}

\begin{titlepage}
  \thispagestyle{empty}
  \centerline {Санкт-Петербургский политехнический университет}
  \centerline { им. Петра Великого}
  \centerline { }
  \centerline {Институт прикладной математики и механики} 
  \centerline {Кафедра "Прикладная математика"}
  \vfill
  \centerline{\textbf{Отчёт}}
  \centerline{\textbf{по лабораторной работе №2}}
  \centerline{\textbf{по дисциплине}}
  \centerline{\textbf{"Математическая статистика"}}
  \vfill
  \hfill
  \begin{minipage}{0.45\textwidth}
  Выполнил студент:\\
  Дроздова Дарья Александровна\\
  группа: 3630102/80401 \\
  \\
  Проверил:\\
  к.ф.-м.н., доцент \\
  Баженов Александр Николаевич
  \end{minipage}
  \vfill
  \centerline {Санкт-Петербург}   
  \centerline {2021 г.}  
\end{titlepage}

\newpage
\setcounter{page}{2}
\tableofcontents

\newpage
\listoftables

\newpage
\section{Постановка задачи}
  Для 5 распределений:
  \begin{itemize}
    \item Нормальное распределение: $N(x,0,1)$
    \item Распределение Коши: $C(x,0,1)$
    \item Распределенеие Лапласа: $L(x,0,\sqrt{2})$
    \item Распределение Пуассона: $P(k,10)$
    \item Равномерное распределение: $U(x,-\sqrt{3}, \sqrt{3})$
  \end{itemize}
  Сгенерировать выборки мощностью 10, 100 и 1000 элементов. Для каждой выборки вычислить статистические положения данных: 
  \begin{itemize}
    \item Выборочное среднее - $\overline{x}$
    \item Выборочная медиана - $med$ $x$
    \item Полусумма экстремальных выборочных элементов - $z_R$
    \item Полусумма квартилей - $z_Q$
    \item Усеченое среднее - $z_{tr}$
  \end{itemize}
  Повторить вычисления 1000 раз для каждой выборки и найти среднее характеристик положения и их квадратов:
  \begin{equation}
    E(z)=\overline{z}
    \label{eq: 1}
  \end{equation}
  Вычислить оценку дисперсии по формуле:
  \begin{equation}
    D(z)=\overline{z^2} - \overline{z}^2
    \label{eq: 2}
  \end{equation}
  Полученные данные представить в виде таблиц
  
\newpage
\section{Теория}
\subsection{Выриационный ряд}
  Вариационный ряд - последовательность элементов выборки, расположенных в неубывающем порядке. Одинаковые элементы повторяются. Запись вариационного ряда: $x_{(1)}, x_{(2)}, ...,x_{(n)}$
\subsection{Выборочное среднее}
 Выборочным средним называется среднее арифметическое значений вариант в выборке, определяется по формуле:
 \begin{equation}
 \overline{x} = \frac{1}{n}\sum_{i=1}^{n}x_i
 \label{eq: 3}
 \end{equation}
 Выборочное среднее используется для статистической оценки математического ожидания исследуемой случайной величины.
\subsection{Выборочная медиана}
  Выборочная медиана определяется по формуле:
  \begin{equation}
  med \; x=\left \{\ 
  \begin{array}{rcl}
  x_{(l+1)}& \qquad \mbox{при} \; n=2l+1 \\
  \frac{x_{(l)}+x_{(l+1)}}{2}& \qquad \mbox{при} \; n=2l
  \end{array}\right.
  \label{eq: 4}
  \end{equation}
\subsection{Полусумма экстремальных выборочных элементов}
  Полусумма экстремальных выборочных элементов определяется по формуле:
  \begin{equation}
  z_R=\frac{x_{(1)}-x_{(n)}}{2}
  \label{eq: 5}
  \end{equation}
\subsection{Полусумма квартилей}
  Для вычисления полусуммы квартилей необходимо для начала вычислить выборочную квартиль $z_p$ порядка $p$, которая определяется по формуле
  \begin{equation}
   z_p=\left \{\ 
  \begin{array}{rcl}
  x_{([np]+1)}& \qquad \mbox{при} \; np \; \mbox{дробном} \\
  x_{(np)}& \qquad \mbox{при} \; np \; \mbox{целом}
  \end{array}\right.
  \label{eq: 6}
  \end{equation}
  Полусумма квартилей определяется по формуле
  \begin{equation}
  z_Q=\frac{z_{\frac{1}{4}} + z_{\frac{3}{4}}}{2}
  \label{eq: 7}
  \end{equation}
\subsection{Усечённое среднее}
  Формула усечённого среднего:
  \begin{equation}
  z_{tr} = \frac{1}{n-2r}\sum_{t=r+1}^{n-r}x_{(i)}, \; r\approx\frac{n}{4}
  \label{eq: 8}
  \end{equation}

\newpage
\section{Реализация}
Лабораторная работа выполнена на языке программирования Python в среде разработки PyCharm с использованием библиотек: numpy, skipy для построения выборок.
\\
\\
Код программы расположен в репозитории GitHub по ссылке: \url{https://github.com/Drozdova-Daria/Math_Stat_Lab2}

\newpage
\section{Результаты}
\subsection{Нормальное распределение}

\begin{table}[hb]
\begin{center}
\begin{tabular}{|c|c|c|c|c|c|}
\hline
\multicolumn{6}{|c|}{Нормальное распределение $n=10$} \\ 
\hline
  & $\overline{x}$ (\ref{eq: 3}) & $med \; x$ (\ref{eq: 4}) & $z_R$ (\ref{eq: 5}) & $z_Q$ (\ref{eq: 7}) & $z_{tr}$ (\ref{eq: 8}) \\ 
\hline
$E(z)$ (\ref{eq: 1}) & -0.0118 & -0.00759 & -0.01094 & 0.16761 & -0.09851\\ 
\hline
$D(z)$ (\ref{eq: 2}) & 0.09598 & 0.14792 & 0.17957 & 0.10526 & 0.08566\\ 
\hline
$E(z)-\sqrt{D(z)}$ & -0.3216 & -0.3922 & -0.4347 & -0.15682 & -0.39118\\ 
\hline
$E(z)+\sqrt{D(z)}$ & 0.29801 & 0.37702 & 0.41282 & 0.49205 & 0.19417\\ 
\hline
$Estimation (\widehat{E}(x))$ & 0 & 0 & 0 & 0 & 0 \\
\hline
\end{tabular} 
\caption{Нормальное распределение для $n=10$}
\end{center} 
\end{table} 

\begin{table}[hb]
\begin{center}
\begin{tabular}{|c|c|c|c|c|c|}
\hline
\multicolumn{6}{|c|}{Нормальное распределение $n=100$} \\ 
\hline
  & $\overline{x}$ (\ref{eq: 3}) & $med \; x$ (\ref{eq: 4}) & $z_R$ (\ref{eq: 5}) & $z_Q$ (\ref{eq: 7}) & $z_{tr}$ (\ref{eq: 8}) \\ 
\hline
$E(z)$ (\ref{eq: 1}) & 0.00397 & 0.0027 & 0.00938 & -0.01152 & -0.00926\\ 
\hline
$D(z)$ (\ref{eq: 2}) & 0.01043 & 0.01481 & 0.09061 & 0.01278 & 0.01146\\ 
\hline
$E(z)-\sqrt{D(z)}$ & -0.09818 & -0.11901 & -0.29163 & -0.12458 & -0.11633\\ 
\hline
$E(z)+\sqrt{D(z)}$ & 0.10612 & 0.12441 & 0.3104 & 0.10155 & 0.09781\\ 
\hline
$Estimation (\widehat{E}(x))$ & 0 & 0 & 0 & 0 & 0 \\
\hline
\end{tabular} 
\caption{Нормальное распределение для $n=100$}
\end{center} 
\end{table} 

\begin{table}[hb]
\begin{center}
\begin{tabular}{|c|c|c|c|c|c|}
\hline
\multicolumn{6}{|c|}{Нормальное распределение $n=1000$} \\ 
\hline
  & $\overline{x}$ (\ref{eq: 3}) & $med \; x$ (\ref{eq: 4}) & $z_R$ (\ref{eq: 5}) & $z_Q$ (\ref{eq: 7}) & $z_{tr}$ (\ref{eq: 8}) \\ 
\hline
$E(z)$ (\ref{eq: 1}) & 0.00133 & -0.00023 & 0.0053 & -0.00156 & -0.00237\\ 
\hline
$D(z)$ (\ref{eq: 2}) & 0.00099 & 0.00156 & 0.05938 & 0.00118 & 0.00115\\ 
\hline
$E(z)-\sqrt{D(z)}$ & -0.03015 & -0.03972 & -0.23838 & -0.03596 & -0.03622\\ 
\hline
$E(z)+\sqrt{D(z)}$ & 0.03281 & 0.03926 & 0.24899 & 0.03284 & 0.03148\\ 
\hline
$Estimation (\widehat{E}(x))$ & 0 & 0 & 0 & 0 & 0 \\
\hline
\end{tabular} 
\caption{Нормальное распределение для $n=1000$}
\end{center} 
\end{table} 

\newpage
\subsection{Распределение Коши}

\begin{table}[hb]
\begin{center}
\begin{tabular}{|c|c|c|c|c|c|}
\hline
\multicolumn{6}{|c|}{Распределение Коши $n=10$} \\ 
\hline
  & $\overline{x}$ (\ref{eq: 3}) & $med \; x$ (\ref{eq: 4}) & $z_R$ (\ref{eq: 5}) & $z_Q$ (\ref{eq: 7}) & $z_{tr}$ (\ref{eq: 8}) \\ 
\hline
$E(z)$ (\ref{eq: 1}) & 53.96927 & 0.00207 & -21.76878 & 0.91222 & -0.20704\\ 
\hline
$D(z)$ (\ref{eq: 2}) & 2958572.88899 & 0.32505 & 430781.06109 & 9.46766 & 0.31104\\ 
\hline
$E(z)-\sqrt{D(z)}$ & -1666.08099 & -0.56806 & -678.10791 & -2.16473 & -0.76475\\ 
\hline
$E(z)+\sqrt{D(z)}$ & 1774.01953 & 0.57221 & 634.57036 & 3.98918 & 0.35066\\ 
\hline
$Estimation (\widehat{E}(x))$ & - & 0 & - & 0 & 0 \\
\hline
\end{tabular} 
\caption{Распределение Коши для $n=10$}
\end{center} 
\end{table} 

\begin{table}[hb]
\begin{center}
\begin{tabular}{|c|c|c|c|c|c|}
\hline
\multicolumn{6}{|c|}{Распределение Коши $n=100$} \\ 
\hline
  & $\overline{x}$ (\ref{eq: 3}) & $med \; x$ (\ref{eq: 4}) & $z_R$ (\ref{eq: 5}) & $z_Q$ (\ref{eq: 7}) & $z_{tr}$ (\ref{eq: 8}) \\ 
\hline
$E(z)$ (\ref{eq: 1}) & 0.37391 & -0.00088 & -693.82325 & -0.02931 & -0.01629\\ 
\hline
$D(z)$ (\ref{eq: 2}) & 3256.86648 & 0.02583 & 363891798.50593 & 0.05351 & 0.02571\\ 
\hline
$E(z)-\sqrt{D(z)}$ & -56.69506 & -0.16159 & -19769.77142 & -0.26064 & -0.17664\\ 
\hline
$E(z)+\sqrt{D(z)}$ & 57.44287 & 0.15983 & 18382.12492 & 0.20202 & 0.14406\\ 
\hline
$Estimation (\widehat{E}(x))$ & 0 & 0 & - & 0 & 0 \\
\hline
\end{tabular} 
\caption{Распределение Коши для $n=100$}
\end{center} 
\end{table} 

\begin{table}[hb]
\begin{center}
\begin{tabular}{|c|c|c|c|c|c|}
\hline
\multicolumn{6}{|c|}{Распределение Коши $n=1000$} \\ 
\hline
  & $\overline{x}$ (\ref{eq: 3}) & $med \; x$ (\ref{eq: 4}) & $z_R$ (\ref{eq: 5}) & $z_Q$ (\ref{eq: 7}) & $z_{tr}$ (\ref{eq: 8}) \\ 
\hline
$E(z)$ (\ref{eq: 1}) & -1.12414 & 0.00024 & 1009.9699 & -0.00099 & -0.0025\\ 
\hline
$D(z)$ (\ref{eq: 2}) & 556.81629 & 0.00257 & 2073280094.77284 & 0.00483 & 0.00256\\ 
\hline
$E(z)-\sqrt{D(z)}$ & -24.72109 & -0.05044 & -44523.31567 & -0.07052 & -0.05308\\ 
\hline
$E(z)+\sqrt{D(z)}$ & 22.47282 & 0.05093 & 46543.25547 & 0.06853 & 0.04808\\ 
\hline
$Estimation (\widehat{E}(x))$ & - & 0 & - & 0 & 0 \\
\hline
\end{tabular} 
\caption{Распределение Коши для $n=1000$}
\end{center} 
\end{table} 

\newpage
\subsection{Распределение Лапласа}

\begin{table}[hb]
\begin{center}
\begin{tabular}{|c|c|c|c|c|c|}
\hline
\multicolumn{6}{|c|}{Распределение Лапласа $n=10$} \\ 
\hline
  & $\overline{x}$ (\ref{eq: 3}) & $med \; x$ (\ref{eq: 4}) & $z_R$ (\ref{eq: 5}) & $z_Q$ (\ref{eq: 7}) & $z_{tr}$ (\ref{eq: 8}) \\ 
\hline
$E(z)$ (\ref{eq: 1}) & -0.0186 & -0.01109 & 0.01204 & 0.36812 & -0.19812\\ 
\hline
$D(z)$ (\ref{eq: 2}) & 0.37199 & 0.27815 & 1.80396 & 0.53026 & 0.20742\\ 
\hline
$E(z)-\sqrt{D(z)}$ & -0.62851 & -0.53848 & -1.33108 & -0.36007 & -0.65355\\ 
\hline
$E(z)+\sqrt{D(z)}$ & 0.59131 & 0.51631 & 1.35515 & 1.09631 & 0.25731\\ 
\hline
$Estimation (\widehat{E}(x))$ & 0 & 0 & 0 & 0 & 0 \\
\hline
\end{tabular} 
\caption{Распределение Лапласа для $n=10$}
\end{center} 
\end{table} 

\begin{table}[hb]
\begin{center}
\begin{tabular}{|c|c|c|c|c|c|}
\hline
\multicolumn{6}{|c|}{Распределение Лапласа $n=100$} \\ 
\hline
  & $\overline{x}$ (\ref{eq: 3}) & $med \; x$ (\ref{eq: 4}) & $z_R$ (\ref{eq: 5}) & $z_Q$ (\ref{eq: 7}) & $z_{tr}$ (\ref{eq: 8}) \\ 
\hline
$E(z)$ (\ref{eq: 1}) & 0.00737 & 0.00479 & 0.0185 & -0.03599 & -0.01481\\ 
\hline
$D(z)$ (\ref{eq: 2}) & 0.03914 & 0.02204 & 1.72822 & 0.04137 & 0.02269\\ 
\hline
$E(z)-\sqrt{D(z)}$ & -0.19047 & -0.14367 & -1.29612 & -0.23938 & -0.16544\\ 
\hline
$E(z)+\sqrt{D(z)}$ & 0.20522 & 0.15326 & 1.33311 & 0.1674 & 0.13583\\ 
\hline
$Estimation (\widehat{E}(x))$ & 0 & 0 & 0 & 0 & 0 \\
\hline
\end{tabular} 
\caption{Распределение Лапласа для $n=100$}
\end{center} 
\end{table} 

\begin{table}[hb]
\begin{center}
\begin{tabular}{|c|c|c|c|c|c|}
\hline
\multicolumn{6}{|c|}{Распределение Лапласа $n=1000$} \\ 
\hline
  & $\overline{x}$ (\ref{eq: 3}) & $med \; x$ (\ref{eq: 4}) & $z_R$ (\ref{eq: 5}) & $z_Q$ (\ref{eq: 7}) & $z_{tr}$ (\ref{eq: 8}) \\ 
\hline
$E(z)$ (\ref{eq: 1}) & -0.00336 & -0.00087 & -0.0182 & -0.00288 & -0.00256\\ 
\hline
$D(z)$ (\ref{eq: 2}) & 0.00386 & 0.00209 & 1.79541 & 0.0041 & 0.00263\\ 
\hline
$E(z)-\sqrt{D(z)}$ & -0.06551 & -0.04658 & -1.35813 & -0.06689 & -0.05383\\ 
\hline
$E(z)+\sqrt{D(z)}$ & 0.0588 & 0.04483 & 1.32173 & 0.06113 & 0.04872\\ 
\hline
$Estimation (\widehat{E}(x))$ & 0 & 0 & 0 & 0 & 0 \\
\hline
\end{tabular} 
\caption{Распределение Лапласа для $n=1000$}
\end{center} 
\end{table} 

\newpage
\subsection{Распределение Пуассона}
\begin{table}[hb]
\begin{center}
\begin{tabular}{|c|c|c|c|c|c|}
\hline
\multicolumn{6}{|c|}{Распределение Пуассона $n=10$} \\ 
\hline
  & $\overline{x}$ (\ref{eq: 3}) & $med \; x$ (\ref{eq: 4}) & $z_R$ (\ref{eq: 5}) & $z_Q$ (\ref{eq: 7}) & $z_{tr}$ (\ref{eq: 8}) \\ 
\hline
$E(z)$ (\ref{eq: 1}) & 9.9931 & 9.903 & 10.3 & 10.54 & 7.94167\\ 
\hline
$D(z)$ (\ref{eq: 2}) & 1.07968 & 1.32959 & 1.8615 & 1.2404 & 0.80438\\ 
\hline
$E(z)-\sqrt{D(z)}$ & 8.95402 & 8.74992 & 8.93563 & 9.42627 & 7.0448\\ 
\hline
$E(z)+\sqrt{D(z)}$ & 11.03218 & 11.05608 & 11.66437 & 11.65373 & 8.83854\\ 
\hline
$Estimation (\widehat{E}(x))$ & $9^{+1}_{-1}$ & $9^{+1}_{-1}$ & $10^{+1}_{-1}$ & $10^{+1}_{-1}$ & $9^{+1}_{-1}$ \\
\hline 
\end{tabular} 
\caption{Распределение Пуассона для $n=10$}
\end{center} 
\end{table} 

\begin{table}[hb]
\begin{center}
\begin{tabular}{|c|c|c|c|c|c|}
\hline
\multicolumn{6}{|c|}{Распределение Пуассона $n=100$} \\ 
\hline
  & $\overline{x}$ (\ref{eq: 3}) & $med \; x$ (\ref{eq: 4}) & $z_R$ (\ref{eq: 5}) & $z_Q$ (\ref{eq: 7}) & $z_{tr}$ (\ref{eq: 8}) \\ 
\hline
$E(z)$ (\ref{eq: 1}) & 10.01207 & 9.8545 & 10.956 & 9.8775 & 9.61274\\ 
\hline
$D(z)$ (\ref{eq: 2}) & 0.10186 & 0.22908 & 0.99356 & 0.16274 & 0.12053\\ 
\hline
$E(z)-\sqrt{D(z)}$ & 9.69291 & 9.37588 & 9.95922 & 9.47408 & 9.26557\\ 
\hline
$E(z)+\sqrt{D(z)}$ & 10.33123 & 10.33312 & 11.95278 & 10.28092 & 9.95991\\ 
\hline
$Estimation (\widehat{E}(x))$ & $10^{+1}_{-1}$ & $10^{+1}_{-1}$ & $10^{+1}_{-1}$ & $9^{+1}_{-1}$ & $9^{+1}_{-1}$ \\
\hline 
\end{tabular} 
\caption{Распределение Пуассона для $n=100$}
\end{center} 
\end{table} 

\begin{table}[hb]
\begin{center}
\begin{tabular}{|c|c|c|c|c|c|}
\hline
\multicolumn{6}{|c|}{Распределение Пуассона $n=1000$} \\ 
\hline
  & $\overline{x}$ (\ref{eq: 3}) & $med \; x$ (\ref{eq: 4}) & $z_R$ (\ref{eq: 5}) & $z_Q$ (\ref{eq: 7}) & $z_{tr}$ (\ref{eq: 8}) \\ 
\hline
$E(z)$ (\ref{eq: 1}) & 10.00362 & 9.9975 & 11.6755 & 9.993 & 9.8335\\ 
\hline
$D(z)$ (\ref{eq: 2}) & 0.01011 & 0.00224 & 0.70345 & 0.00345 & 0.01054\\ 
\hline
$E(z)-\sqrt{D(z)}$ & 9.90305 & 9.95013 & 10.83678 & 9.93425 & 9.73083\\ 
\hline
$E(z)+\sqrt{D(z)}$ & 10.10419 & 10.04487 & 12.51422 & 10.05175 & 9.93618\\ 
\hline
$Estimation (\widehat{E}(x))$ & $10^{+1}_{-1}$ & $9^{+1}_{-1}$ & $10^{+1}_{-1}$ & $9^{+1}_{-1}$ & $9^{+1}_{-1}$ \\
\hline 
\end{tabular} 
\caption{Распределение Пуассона для $n=1000$}
\end{center} 
\end{table} 

\newpage
\subsection{Равномерное распределение}

\begin{table}[hb]
\begin{center}
\begin{tabular}{|c|c|c|c|c|c|}
\hline
\multicolumn{6}{|c|}{Равномерное распределение $n=10$} \\ 
\hline
  & $\overline{x}$ (\ref{eq: 3}) & $med \; x$ (\ref{eq: 4}) & $z_R$ (\ref{eq: 5}) & $z_Q$ (\ref{eq: 7}) & $z_{tr}$ (\ref{eq: 8}) \\ 
\hline
$E(z)$ (\ref{eq: 1}) & -0.87033 & -0.87319 & -0.86208 & -0.77866 & -0.77771\\ 
\hline
$D(z)$ (\ref{eq: 2}) & 0.02406 & 0.05447 & 0.01038 & 0.02797 & 0.0331\\ 
\hline
$E(z)-\sqrt{D(z)}$ & -1.02543 & -1.10659 & -0.96396 & -0.9459 & -0.95963\\ 
\hline
$E(z)+\sqrt{D(z)}$ & -0.71523 & -0.6398 & -0.7602 & -0.61143 & -0.59578\\ 
\hline
$Estimation (\widehat{E}(x))$ & 0 & 0 & 0 & 0 & 0 \\
\hline
\end{tabular} 
\caption{Равномерное распределение для $n=10$}
\end{center} 
\end{table} 

\begin{table}[hb]
\begin{center}
\begin{tabular}{|c|c|c|c|c|c|}
\hline
\multicolumn{6}{|c|}{Равномерное распределение $n=100$} \\ 
\hline
  & $\overline{x}$ (\ref{eq: 3}) & $med \; x$ (\ref{eq: 4}) & $z_R$ (\ref{eq: 5}) & $z_Q$ (\ref{eq: 7}) & $z_{tr}$ (\ref{eq: 8}) \\ 
\hline
$E(z)$ (\ref{eq: 1}) & -0.86591 & -0.8664 & -0.86522 & -0.87295 & -0.85769\\ 
\hline
$D(z)$ (\ref{eq: 2}) & 0.00247 & 0.00666 & 0.00014 & 0.00382 & 0.0049\\ 
\hline
$E(z)-\sqrt{D(z)}$ & -0.91559 & -0.94802 & -0.87716 & -0.93473 & -0.92769\\ 
\hline
$E(z)+\sqrt{D(z)}$ & -0.81624 & -0.78479 & -0.85327 & -0.81117 & -0.78769\\ 
\hline
$Estimation (\widehat{E}(x))$ & 0 & 0 & 0 & 0 & 0 \\
\hline
\end{tabular} 
\caption{Равномерное распределение для $n=100$}
\end{center} 
\end{table} 

\begin{table}[hb]
\begin{center}
\begin{tabular}{|c|c|c|c|c|c|}
\hline
\multicolumn{6}{|c|}{Равномерное распределение $n=1000$} \\ 
\hline
  & $\overline{x}$ (\ref{eq: 3}) & $med \; x$ (\ref{eq: 4}) & $z_R$ (\ref{eq: 5}) & $z_Q$ (\ref{eq: 7}) & $z_{tr}$ (\ref{eq: 8}) \\ 
\hline
$E(z)$ (\ref{eq: 1}) & -0.86621 & -0.86581 & -0.86603 & -0.86715 & -0.86498\\ 
\hline
$D(z)$ (\ref{eq: 2}) & 0.00025 & 0.00075 & 0.000002 & 0.00034 & 0.0005\\ 
\hline
$E(z)-\sqrt{D(z)}$ & -0.882 & -0.89328 & -0.86725 & -0.88547 & -0.88724\\ 
\hline
$E(z)+\sqrt{D(z)}$ & -0.85041 & -0.83834 & -0.8648 & -0.84883 & -0.84272\\ 
\hline
$Estimation (\widehat{E}(x))$ & 0 & 0 & 0 & 0 & 0 \\
\hline
\end{tabular} 
\caption{Равномерное распределение для $n=1000$}
\end{center} 
\end{table} 

\newpage
\section{Обсуждение}
По результатам полученных данных можно заметить, что дисперсия характеристик распределения для распределения Коши дает большие значения даже при увеличении можщности выборки, это может быть связано с выбросами, которые наблюдались на гистограмме для данного распределения при выполнении лабораторной работы № 1.

\newpage
\begin{thebibliography}{4}
\addcontentsline{toc}{section}{\bibname}
\bibitem{cauchy}
Документация бибилиотеки scipy.stats.cauchy. 
\\ URL: https://docs.scipy.org/doc/scipy/reference/generated/scipy.stats.cauchy.html
\bibitem{laplace}
Документация бибилиотеки scipy.stats.laplace. 
\\ URL: https://docs.scipy.org/doc/scipy/reference/generated/scipy.stats.laplace.html
\bibitem{poisson}
Документация бибилиотеки scipy.stats.poisson.
\\ URL: https://docs.scipy.org/doc/scipy/reference/generated/scipy.stats.poisson.html
\bibitem{uniform}
Документация бибилиотеки scipy.stats.uniform.
\\ URL: https://docs.scipy.org/doc/scipy/reference/generated/scipy.stats.uniform.html
\end{thebibliography}



\end{document}
